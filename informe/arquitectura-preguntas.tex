\subsection{Aspectos particularmente importantes y cómo los resolvimos}

\subsubsection{¿Cómo se cumplen los requerimientos de seguridad respecto de los servicios para las empresas enfocadas en los grandes datos?}

Utilizamos varias tácticas para cumplir con los requerimientos de seguridad.

Primero tenemos todos los datos de las bases de datos encriptados, tanto de la base de datos de Información de Usuarios como de Acciones de Usuarios. Esto permite que, aunque haya una intrusión en el sistema y las bases de datos se vean expuestas, el atacante no podrá obtener ningún tipo de información sobre los usuarios. Esto es una táctica de resistencia a ataques.

Otra táctica que usamos es la desanonimización de los datos de los usuarios. Esto, desde algún punto de vista, puede verse como la táctica de separación de entidades. Esto nos permite que, por ejemplo, si hay alguna intrusión del lado de la empresa de grandes datos, no se pueda recuperar la información personal de los usuarios.

Por último, todos los enlaces entre la empresa de grandes datos y nuestro sistema son a través de una conexión segura como la que está explicada en los diagramas, utilizando por ejemplo TLS, que nos brinda autenticación y no repudio, que son dos características muy deseables para el tipo de conexión que queremos establecer.

\subsubsection{¿Cómo es la interacción con los servicios externos? (API de moderación, Aerolíneas, Cámara de Hoteles) ¿Hay disponibilidad? ¿Seguridad?}

La interacción con los servicios externos es siempre mediante manejadores de esos servicios externos que nos abstraen y nos permiten reemplazarlos fácilmente.

En cuanto a disponibilidad, implementamos un Ping Requester, que cada 10 minutos envía pings a los servicios para verificar que estén disponibles y funcionando correctamente. Si no lo están, se envía un mail al on-call del equipo, reportando el problema. Esta táctica es de detección de errores, y permite disminuir el downtime.

Por otra parte, en servicios como las publicidades, implementamos caches de publicidades, que ante un problema de conexión con la API de Aerolíneas Argentinas que nos provee de publicidades, carga una publicidad de la cache de publicidades y la muestra.

En cuanto a seguridad, implementamos el conector seguro para conectarnos a todos los servicios externos. Estos conectores usan TLS, que brinda integridad, autenticación y no repudio, todo lo deseable en una conexión segura. Esto nos permite resistir muy bien a los ataques. Además, nunca se envía información personal de los usuarios. Cada vez que información de usuarios va a ser envíada fuera del sistema, es anonimizada (se le quitan los datos personales y se la deja
inidentificable a una persona física) y encriptada.

\subsubsection{¿Cómo se logra perfomance para responder las búsquedas de los usuarios?}

La performance se logra utilizando variadas tácticas.

La táctica más simple es la de replicación. Tenemos nuestro servidor replicado y tenemos un manejador de requests entre la UI y el server side, que actúa como distribuidor de requests y balanceador de carga.

Además, dentro del Manejador de Eventos, en la sección que concierne a la búsqueda de eventos, tenemos múltiples caches que permiten obtener la información a muchisima más velocidad que haciendo la búsqueda de cero. 

Además, tenemos replicadas nuestras bases de datos, lo cual permite atender varias requests de lectura al mismo tiempo sin problemas.


